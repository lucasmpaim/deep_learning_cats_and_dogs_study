% ----------------------------------------------------------
% Introdução
% ----------------------------------------------------------
\section{Introdução}

Segundo \citeonline{handsOnML}, apesar de os computadores já terem conseguido feitos notáveis como vencer o campeão mundial de xadrez em 1996, eles ainda não conseguiam realizar tarefas triviais, como detectar um animal em uma foto, o que para nós humanos é algo extremamente trivial, isto ainda segundo \citeonline{handsOnML} é devido ao nosso cérebro enviar para nosso subconsciente apenas informações de auto-nível, as redes neurais convolucionais (CNN), surgiram do estudo de nosso córtex visual e essas redes tem sido utilizadas como o principal meio para detecção de imagens desde então.

Neste trabalho será abordado algumas diferentes técnicas para a resolução do dataset cats vs dogs já pré-processada disponibilizada pelo Google na página do Tensorflow, onde foram removidas cerca de 1738 imagens corrompidas. 