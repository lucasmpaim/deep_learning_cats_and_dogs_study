\subsection{Model from Scratch}

O modelo criado do zero, foi montado baseado no modelo proposto pelo \citeonline{tensorflow-cats-vs-dogs} e em algumas configurações da Xception, sua arquitetura proposta foi:


\begin{verbatim}
_________________________________________________________________
Layer (type)                 Output Shape              Param #   
=================================================================
conv2d_3 (Conv2D)            (None, 299, 299, 16)      448       
_________________________________________________________________
max_pooling2d_3 (MaxPooling2 (None, 149, 149, 16)      0         
_________________________________________________________________
conv2d_4 (Conv2D)            (None, 149, 149, 32)      4640      
_________________________________________________________________
max_pooling2d_4 (MaxPooling2 (None, 74, 74, 32)        0         
_________________________________________________________________
conv2d_5 (Conv2D)            (None, 74, 74, 64)        18496     
_________________________________________________________________
max_pooling2d_5 (MaxPooling2 (None, 37, 37, 64)        0         
_________________________________________________________________
flatten_1 (Flatten)          (None, 87616)             0         
_________________________________________________________________
dropout_1 (Dropout)          (None, 87616)             0         
_________________________________________________________________
dense_2 (Dense)              (None, 512)               44859904  
_________________________________________________________________
dense_3 (Dense)              (None, 1)                 513       
=================================================================
Total params: 44,884,001
Trainable params: 44,884,001
Non-trainable params: 0
_________________________________________________________________
\end{verbatim}

\subsubsection{Resultados do Treinamento}

O resultado preliminar foi de 85,1\% de acurácia na base de teste e 68.8\% na base de validação.


\begin{verbatim}
loss: 0.3271 - accuracy: 0.8510 - val_loss: 0.6158 - val_accuracy: 0.6886
\end{verbatim}


É um resultado decepcionante considerando que um modelo shallow consiguiu 97\% de acurácia, porém, não foram aplicados meios de se melhorar esta rede para esta quantidade de imagens, que seria aplicar técnicas de aumento de dados, ou seja, aplicar transformações na imagem para que cada imagem gere N outras imagens, o que ajuda no treinamento da rede.

A matrix de confusão pode ser consultada na: \autoref{cm-from-scratch}, 