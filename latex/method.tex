% ----------------------------------------------------------
% Seção de explicações
% ----------------------------------------------------------
\section{Método}

O dataset utilizado por este trabalho, foi a "cats vs dogs" disponibilizada pelo Google em sua página da tensorflow, esta base contém 3 mil imagens de gatos e cachorros e seu propósito é ensinar a computadores a classificar fotos destes dois animais.
A base está dividida em treinamento e validação, onde respectivamente temos, 2 mil imagens para treinamento e mil imagens para validação, ou seja, aproximadamente 30\% da base voltada para validação.

As técnicas utilizadas para este trabalho são:

 \begin{description}
	\item[Transfer-learning baseada em extrator de caracterísicas] é utilizada uma rede previamente treinada como extrator de caracterísicas, e estas são utilizadas para treinar modelos razos, como: SVM, KNN, etc.
	\item[Transfer-learning baseado em fine-tuning] é utilizada uma rede previamente treinada, em que as camadas convolucionais do modelo são congeladas, e as camadas densas são retreinadas para o novo problema.
	\item[From scratch] Criar e treinar um modelo a partir de um novo projeto, ou seja, sem reuilizar redes previamente treinadas.
\end{description}


\subsection{Transfer Learning Shallow}

O modelo pré-treinado utilizado por este trabalho é a Xception que é uma rede treinada utilizando a base \citeonline{imagenet}, que é uma base de com milhares de imagens utilizada e alimentada pela comunidade cientifica.

Foram extraídos as características em arquivos CSV, esta extração resultou em 2048 características para cada uma das imagens, o dataframe de treinamento obteve o seguinte shape:

\begin{verbatim}
	(2000, 2048)
\end{verbatim}

Utilizando a técnica t-SNE, que segundo \citeonline{t-SNE}, é uma técnica de redução de dimensionalidade estocástica, ou seja, utilizando está técnica é possível reduzir um vetor de tamanho \(N\) para um de tamanho \(X\), onde \(X < N\), foi realizada a redução das bases de características e apresentadas gráficamente, para assim analisar o quão bem distribuídos estão as classes, estes gráficos podem ser conferinos na \autoref{tsne-train-transfer-shallow} e \autoref{tsne-val-transfer-shallow}

É possível notar estas características conseguem separar quase que perfeitamente as duas classes (gatos e cães), já na visão reduzida que temos na plotagem do t-SNE, com isso não é necessário um modelo muito robusto para que se obtenha resultados satisfatórios. 

O modelo shallow escolhido para este trabalho é o Naive Bayes.

Segundo, \citeonline{naive-bayes-britto}, o teorema de Bayes consiste em uma abordagem probabilística para aprendizagem, calculando assim a probabilidade de uma característica \(X\) de um vetor de prever determinada classe.

A técnica de Naive Bayes, assume que todas as caractéristicas no vetor são independentes, por isso, é chamado de Naive (do inglês Ingênuo), pois determinadas caracterísicas podem estar relacionadas entre si, o que faz com que essa técnica tenha um resultado inferior a ténica de Redes Bayesianas que consideram que os atributos podem ou não ser relacionados, isto é, a característica \(X\) pode ou não estar diretamente relacionada a característica \(Y\).

\subsubsection{Resultados do Treinamento}

Como esperado, o resultado foi extremamente satisfatório, conseguindo até 97,6\% de acurácia.


\begin{verbatim}
	Acurácia Naive Bayes Validação: 0.976
\end{verbatim}

A matrix de confusão pode ser consultada na: \autoref{cm-naive-bayes}.
\subsection{Transfer Learning - Fine Tunning}


\subsubsection{Resultados do Treinamento}
\subsection{Model from Scratch}

O modelo criado do zero, foi montado baseado no modelo proposto pelo \citeonline{tensorflow-cats-vs-dogs} e em algumas configurações da Xception, sua arquitetura proposta foi:


\begin{verbatim}
_________________________________________________________________
Layer (type)                 Output Shape              Param #   
=================================================================
conv2d_3 (Conv2D)            (None, 299, 299, 16)      448       
_________________________________________________________________
max_pooling2d_3 (MaxPooling2 (None, 149, 149, 16)      0         
_________________________________________________________________
conv2d_4 (Conv2D)            (None, 149, 149, 32)      4640      
_________________________________________________________________
max_pooling2d_4 (MaxPooling2 (None, 74, 74, 32)        0         
_________________________________________________________________
conv2d_5 (Conv2D)            (None, 74, 74, 64)        18496     
_________________________________________________________________
max_pooling2d_5 (MaxPooling2 (None, 37, 37, 64)        0         
_________________________________________________________________
flatten_1 (Flatten)          (None, 87616)             0         
_________________________________________________________________
dropout_1 (Dropout)          (None, 87616)             0         
_________________________________________________________________
dense_2 (Dense)              (None, 512)               44859904  
_________________________________________________________________
dense_3 (Dense)              (None, 1)                 513       
=================================================================
Total params: 44,884,001
Trainable params: 44,884,001
Non-trainable params: 0
_________________________________________________________________
\end{verbatim}

\subsubsection{Resultados do Treinamento}

O resultado preliminar foi de 85,1\% de acurácia na base de teste e 68.8\% na base de validação.


\begin{verbatim}
loss: 0.3271 - accuracy: 0.8510 - val_loss: 0.6158 - val_accuracy: 0.6886
\end{verbatim}


É um resultado decepcionante considerando que um modelo shallow consiguiu 97\% de acurácia, porém, não foram aplicados meios de se melhorar esta rede para esta quantidade de imagens, que seria aplicar técnicas de aumento de dados, ou seja, aplicar transformações na imagem para que cada imagem gere N outras imagens, o que ajuda no treinamento da rede.

A matrix de confusão pode ser consultada na: \autoref{cm-from-scratch}, 